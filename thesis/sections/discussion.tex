\section{Discussion}

\subsection{Interpretation of Treatment Effects}
The estimated effect of a 5\% VAT increase leading to a 3.87\% reduction in firm survival rate is both economically and statistically significant. This finding provides compelling evidence that fiscal policy changes can materially influence microeconomic firm behavior, with downstream impacts on employment, investment, and economic dynamism. The causal modeling framework captures not only average treatment effects but also heterogeneous impacts across economic environments, emphasizing the nonlinear and context-dependent nature of tax policy consequences. Particularly in downturns, the amplified adverse effects highlight the fragility of small and medium enterprises to tax shocks.

\subsection{Policy Implications}
The robust evidence of significant negative impacts on firm survival mandates caution among policymakers considering VAT increases as a revenue tool. Given the potential for unintended economic contraction and job losses, the findings recommend exploring alternative fiscal strategies that balance revenue needs with economic resilience. Scenario analyses illustrate that timing and macroeconomic context critically modulate policy outcomes, suggesting adaptive tax policies that account for economic cycles may mitigate harmful effects. Policymakers should also incorporate distributional analyses to safeguard vulnerable sectors and businesses, promoting inclusive economic growth.

\subsection{Comparative Framework Strengths}
The hybrid causal machine learning framework developed integrates advances across time-series forecasting, causal inference, and heterogeneous treatment effect modeling. Combining LSTM networks for baseline economic trend estimation with Double Machine Learning and Causal Forest methods for unbiased and granular causal effect estimation represents a methodological breakthrough over conventional methods. This multi-model integration improves prediction accuracy, reduces bias, and uncovers meaningful heterogeneity, allowing richer policy insights and reducing uncertainty. The ensemble approach also enhances model stability across time and economic regimes, proving scalable and adaptable to complex economic data.

\subsection{Comparison to Traditional Econometric Benchmarks}
Traditional econometric models, while offering interpretable coefficients grounded in economic theory, often rely on strong assumptions such as linearity and exogeneity that may be violated in real-world data. These constraints reduce flexibility in capturing nonlinearities and high-dimensional confounding prevalent in large economic datasets. Machine learning models traditionally excel at prediction but struggle with causal interpretability. Our hybrid approach combines the strengths of both—leveraging machine learning’s ability to model complex patterns while maintaining econometric rigor for causal validity. Empirically, our framework demonstrates superior predictive performance and richer policy-relevant inference than either approach in isolation.

\subsection{Alignment with Economic Theory}
The heterogeneous effects detected align well with economic theories of firm behavior under tax shocks, such as the responsiveness of firm entry and exit rates to cost changes. The greater negative impact during downturns conforms to theories of financial constraint and frictions that exacerbate firm vulnerability under stress. The model’s ability to quantify these nuanced economic relationships provides validation of its substantive grounding and reinforces trust in the policy simulations as economically plausible.

\subsection{Limitations}
Despite these advances, the research faces limitations primarily stemming from data availability and quality. High-resolution micro and macroeconomic datasets integrating firm-level outcomes with detailed tax policy changes remain scarce and often confidential. These constraints limit sample size, temporal coverage, and granularity, potentially biasing causal effect estimates or restricting generalizability. Data accessibility issues also impede replication and extension efforts by other researchers. Furthermore, model performance challenges such as residual autocorrelation and calibration issues underline the need for ongoing methodological refinement and richer datasets capturing economic complexity.

\subsection{Future Directions}
Future research should prioritize enhanced data collection and sharing initiatives to overcome data access challenges, enabling more comprehensive analysis of VAT and other fiscal policies. Methodological innovations incorporating causal discovery and reinforcement learning may further improve dynamic policy evaluation. Expanding the framework to differentiate impacts across heterogeneous firm demographics and geographic regions will also enhance policy targeting capabilities.

Overall, this study demonstrates the promise of integrating modern machine learning and econometric techniques to deliver nuanced, data-driven fiscal policy insights capable of informing resilient and equitable economic governance.
