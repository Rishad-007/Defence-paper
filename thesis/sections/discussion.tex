\section{Discussion}\label{sec:discussion}
This section interprets empirical findings, examines theoretical alignment, and outlines policy implications and limitations.

\subsection{Interpretation of Effects}
The attenuation of medium-run output alongside modest inflation acceleration under \VAT{} adjustments aligns with partial pass-through and transitory consumption smoothing. Heterogeneous amplification during high-inflation regimes suggests interaction effects between tax incidence and expectation anchoring.

\subsection{Policy Implications}
Results indicate that timing \VAT{} adjustments during low-volatility, anchored-expectation periods may mitigate adverse output deviations. Targeted compensatory transfers for vulnerable cohorts could offset regressive expenditure burdens identified in subgroup effect surfaces.

\subsection{Comparative Framework Strengths}
The hybrid design (i) fuses dynamic state extraction with causal heterogeneity modeling, (ii) enhances out-of-sample stability, and (iii) preserves interpretability via modular decomposition. This bridges limitations inherent in purely structural or purely predictive paradigms.

\subsection{Comparison to Traditional Econometric Benchmarks}
Relative to VAR and DiD frameworks, the hybrid model produces narrower forecast error dispersion and higher discriminatory power in subgroup effect surfaces. Unlike reduced-form regressions, latent embeddings capture evolving macro regimes without prespecifying interaction hierarchies.

\subsection{Alignment with Theory}
Observed attenuation patterns are consistent with intertemporal consumption smoothing and partial tax pass-through under nominal rigidities. Amplified responses in high-inflation states echo expectations unanchoring frictions highlighted in modern Phillips curve variants.

\subsection{Limitations}
Limitations include potential residual confounding from latent policy anticipation, finite-sample instability in extreme macro regimes, and abstraction from general equilibrium feedback loops. Model choice introduces dependence on hyperparameter tuning heuristics.

\subsection{Future Extensions}
Promising extensions include: reinforcement learning for sequential policy optimization, structural alignment via semi-parametric DSGE overlays, distributionally robust training objectives, and multi-country transferability studies.
