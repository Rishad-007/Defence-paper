% Introduction section template
\section{Introduction}

% Your introduction goes here.

In the rapidly evolving landscape of economic policy analysis, the integration of advanced computational methods with traditional econometric approaches has become increasingly vital. Policymakers and researchers are confronted with complex, high-dimensional datasets and dynamic economic environments that challenge the limits of conventional analytical tools. As a result, there is a growing need for innovative frameworks that can harness the predictive power of machine learning while preserving the interpretability and theoretical rigor of econometric models.

One of the most consequential areas of economic policy is the design and evaluation of tax interventions, such as changes in value-added tax (VAT) rates. VAT policies have far-reaching implications for government revenue, consumer behavior, business investment, and overall economic growth. However, accurately quantifying the effects of such policies remains a formidable challenge due to the interplay of multiple macroeconomic factors, structural shifts, and heterogeneous responses across sectors and populations.

Recent advances in machine learning, particularly in time-series forecasting and causal inference, offer promising avenues for addressing these challenges. Techniques such as Long Short-Term Memory (LSTM) neural networks and causal forests enable the modeling of complex, nonlinear relationships and the estimation of both average and heterogeneous treatment effects. When combined with robust econometric methods, these tools can provide deeper insights into the mechanisms and outcomes of policy interventions.

This study proposes a hybrid analytical framework that leverages both machine learning and econometric techniques to evaluate the impact of VAT policy changes. By utilizing a comprehensive dataset of macroeconomic indicators and employing rigorous validation strategies, the research aims to enhance the accuracy, reliability, and interpretability of policy impact assessments. The findings are intended to inform policymakers and contribute to the development of more effective, data-driven economic strategies in an increasingly complex world.

\subsection{Background}
Recent studies on fiscal policy and economic recovery \citep{imf2023} have shown the importance of understanding policy impacts in economic decision-making.

\subsection{Research Question}
% Research question content

\subsection{Contributions}
% Contributions content

\subsection{Structure of the Thesis}
% Structure content
