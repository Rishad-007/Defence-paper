\section{Introduction}\label{sec:intro}

Fiscal policy decisions, such as adjusting Value Added Tax (VAT), are among the most influential tools available to governments for managing economic performance and public finance. Yet, determining the true impact of such interventions remains a profound challenge. Policy changes often ripple across multiple layers of the economy---affecting inflation, consumer behavior, income distribution, and macroeconomic stability. Accurately predicting these outcomes, both before and after implementation, is essential for minimizing policy regret and supporting inclusive economic planning.

\subsection{Background \& Motivation}\label{subsec:background}

VAT plays a central role in \textbf{macroeconomic} and \textbf{fiscal management}, particularly in developing and emerging economies. Unlike income taxes, VAT can be applied broadly and collected efficiently, making it a key instrument for raising government revenue, maintaining budgetary discipline, and funding social programs. However, its effects are not neutral: VAT increases may lead to inflationary pressure, reduced consumption, and adverse welfare effects---especially on lower-income populations.

To assess such trade-offs, policymakers have traditionally relied on \textbf{econometric methods} such as Ordinary Least Squares (OLS), Difference-in-Differences (DiD), and Vector Autoregression (VAR). While these tools have been foundational in empirical economics, they suffer from significant limitations:

\begin{itemize}
    \item \textbf{Strong parametric assumptions} (e.g., linearity, homoscedasticity) that may not reflect complex real-world dynamics.
    \item \textbf{Limited capacity} to handle high-dimensional or non-linear confounding structures in observational data.
    \item \textbf{Weak generalization} beyond the observed data, making it difficult to forecast under novel policy conditions.
    \item \textbf{Insufficient granularity}, especially in detecting heterogeneous effects across different socioeconomic groups.
\end{itemize}

These constraints reduce the reliability of counterfactual estimates derived from traditional models and limit their usefulness in policy design, especially in contexts requiring precise targeting and risk minimization.

\subsection{Research Question}\label{subsec:research_question}

\begin{quote}
\textbf{Primary Research Question:}

How can a hybrid causal machine learning framework---combining forecasting, causal estimation, and heterogeneity modeling---enhance counterfactual policy analysis compared to traditional inference?

\textbf{Sub-Questions:}

\begin{enumerate}
    \item How do macroeconomic indicators respond to VAT changes under deep learning-based forecasting models?
    \item How effectively can DoubleML isolate VAT effects while adjusting for high-dimensional confounders?
    \item How does the effect of VAT vary across demographic subgroups, and how can Causal Forests reveal this heterogeneity?
    \item Can the integration of these models reduce expected policy regret and improve the design of future fiscal interventions?
\end{enumerate}
\end{quote}

\subsection{Contributions}\label{subsec:contributions}

This thesis makes the following contributions:

\begin{itemize}
    \item \textbf{A unified hybrid framework} for causal economic policy evaluation, combining diverse modeling paradigms including deep learning and causal inference.
    
    \item \textbf{A modular causal pipeline} consisting of discovery, estimation, validation, and simulation, tailored for robust counterfactual reasoning in fiscal policy.
    
    \item \textbf{Empirical results} derived from macroeconomic time-series data and household-level demographic data.
    
    \item \textbf{Granular treatment effect} estimation, allowing for the discovery of group-specific policy impacts using Causal Forests.
    
    \item \textbf{Comparative analysis}, demonstrating the advantages of this hybrid framework over traditional econometric approaches in flexibility, robustness, and real-world relevance.
\end{itemize}

\subsection{Structure of the Paper}\label{subsec:structure}

The rest of the paper is structured as follows:

\begin{itemize}
    \item \textbf{Section~\ref{sec:litreview}} reviews existing literature on causal inference, machine learning for economics, and counterfactual analysis.
    
    \item \textbf{Section~\ref{sec:data}} details the datasets, data preprocessing, and variable construction.
    
    \item \textbf{Section~\ref{sec:methods}} presents the hybrid methodological framework, outlining each model and integration strategy.
    
    \item \textbf{Section~\ref{sec:results}} reports experimental results, including visualizations and robustness checks.
    
    \item \textbf{Section~\ref{sec:discussion}} discusses key findings, policy implications, and the comparative strengths of this framework.
    
    \item \textbf{Section~\ref{sec:conclusion}} concludes with a summary and directions for future research.
\end{itemize}

Recent advances in machine learning and modern causal inference offer promising avenues to address these gaps. Sequence models such as Long Short-Term Memory (LSTM) networks capture temporal dependencies and regime shifts, while causal forests and related meta-learners enable flexible estimation of heterogeneous treatment effects without imposing restrictive functional forms. When combined with principled econometric structure—through validation, identification strategies, and interpretability diagnostics—these methods can enhance both predictive accuracy and causal robustness.

This thesis responds to these needs by developing a hybrid modeling framework that integrates machine learning forecasting with causal inference pipelines to evaluate the macroeconomic and distributional impacts of VAT policy adjustments. By unifying rigorous identification logic with flexible function approximation, the framework seeks to provide more reliable policy-relevant counterfactuals and to surface heterogeneity essential for equitable and efficient fiscal design. The subsequent sections detail the data architecture, modeling strategy, empirical evaluation, and policy interpretation.


\subsection{Research Question}
% Research question content

\subsection{Contributions}
% Contributions content

\subsection{Structure of the Thesis}
% Structure content
