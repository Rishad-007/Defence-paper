\section{Results}\label{sec:results}
This section presents empirical findings from forecasting performance, heterogeneous effect estimation, feature attribution, and policy scenario simulations.

\subsection{Forecast Performance}
Table~\ref{tab:forecast_performance} reports illustrative error metrics across model classes. The hybrid framework outperforms traditional baselines, particularly on medium-horizon forecasts. A detailed breakdown of rolling-origin forecast distributions will be added in Table~\ref{tab:forecast_horizon_breakdown} (PLACEHOLDER).

\begin{table}[H]
  \centering
  \caption{Illustrative Forecast Performance (Placeholder)}\label{tab:forecast_performance}
  \begin{tabular}{lrrrr}
    \toprule
    Model & MAE & RMSE & MASE & DirAcc \\
    \midrule
    ARIMA & 0.98 & 1.42 & 1.10 & 0.54 \\
    VAR & 0.91 & 1.35 & 1.02 & 0.56 \\
    Static Regression & 1.05 & 1.50 & 1.18 & 0.52 \\
    LSTM & 0.72 & 1.08 & 0.82 & 0.61 \\
    Causal Forest + Lags & 0.85 & 1.20 & 0.97 & 0.58 \\
    Hybrid (LSTM + CF) & 0.66 & 0.99 & 0.75 & 0.64 \\
    \bottomrule
  \end{tabular}
\end{table}

\subsection{Heterogeneous Treatment Effects}
Figure~\ref{fig:heterogeneous_effects} visualizes estimated conditional treatment effects across macro regimes. High-inflation periods exhibit amplified \VAT{} pass-through effects relative to stable-price regimes. Table~\ref{tab:cate_summary} summarizes key quantiles of the heterogeneous effect distribution (placeholder values to be replaced with parsed CSV metrics).

\begin{figure}[H]
  \centering
  \fbox{\parbox{0.75\textwidth}{\centering [Figure Placeholder: Heterogeneous Treatment Effects]\\ \vspace{2cm}}}
  \caption{Estimated Heterogeneous Treatment Effects (Illustrative)}\label{fig:heterogeneous_effects}
\end{figure}

\subsection{Feature Importance}
Global and local importance diagnostics identify inflation momentum, labor market slack, and lagged output gaps as primary moderators of \VAT{} effects. Figure~\ref{fig:feature_importance} summarizes variable influence. Table~\ref{tab:feature_importance_cf} and Table~\ref{tab:feature_importance_dml} present top-ranked predictors from causal forest and DML pipelines respectively (PLACEHOLDER rows—final values to be auto-filled from CSVs).

\begin{figure}[H]
  \centering
  \fbox{\parbox{0.75\textwidth}{\centering [Figure Placeholder: Feature Importance]\\ \vspace{2cm}}}
  \caption{Feature Importance (Illustrative Placeholder)}\label{fig:feature_importance}
\end{figure}

\subsection{Policy Scenario Simulation}
Scenario analysis evaluates a hypothetical \VAT{} rate increase relative to a no-change baseline. Table~\ref{tab:policy_impact_summary} summarizes cumulative impacts over selected horizons. Extended scenario comparisons (shock timing, multi-step adjustments) will be placed in Table~\ref{tab:policy_extended} (PLACEHOLDER) and visualized in Figure~\ref{fig:policy_risk_frontier} (planned).

\begin{figure}[H]
  \centering
  \fbox{\parbox{0.75\textwidth}{\centering [Figure Placeholder: Policy Risk Frontier]\\ \vspace{2cm}}}
  \caption{Policy Risk Frontier Analysis (Illustrative Placeholder)}\label{fig:policy_risk_frontier}
\end{figure}

\begin{table}[H]
  \centering
  \caption{Policy Impact Summary (Illustrative)}\label{tab:policy_impact_summary}
  \begin{tabular}{lrrr}
    \toprule
    Horizon & Output Deviation (pp) & Inflation Impact (pp) & Unemployment Change (pp) \\
    \midrule
    4 Quarters & -0.30 & +0.25 & +0.10 \\
    8 Quarters & -0.55 & +0.40 & +0.18 \\
    12 Quarters & -0.62 & +0.38 & +0.15 \\
    \bottomrule
  \end{tabular}
\end{table}

\subsection{Robustness and Stability}
Alternative model specifications retain effect directionality. Bootstrap intervals narrow for aggregate ATE estimates but widen for tail-regime CATEs, reflecting sparse support. Placebo test rejection frequencies and overlap diagnostics will be summarized in Table~\ref{tab:placebo_overlap} (PLACEHOLDER).

\subsection{Summary}
The hybrid approach enhances forecast precision, uncovers economically plausible heterogeneity, and produces policy effect estimates consistent with theoretical expectations under demand-supply adjustment channels.

% Additional Tables Placeholders (auto-populated in later revision)
\begin{table}[H]
  \centering
  \caption{Forecast Horizon Breakdown (PLACEHOLDER)}\label{tab:forecast_horizon_breakdown}
  \begin{tabular}{lrrrrr}
    	oprule
    Horizon & Model & MAE & RMSE & MAPE & Coverage \\
    \midrule
  1Q & Hybrid & \textemdash{} & \textemdash{} & \textemdash{} & \textemdash{} \\
  4Q & Hybrid & \textemdash{} & \textemdash{} & \textemdash{} & \textemdash{} \\
  8Q & Hybrid & \textemdash{} & \textemdash{} & \textemdash{} & \textemdash{} \\
  12Q & Hybrid & \textemdash{} & \textemdash{} & \textemdash{} & \textemdash{} \\
    \bottomrule
  \end{tabular}
\end{table}

\begin{table}[H]
  \centering
  \caption{CATE Distribution Summary (PLACEHOLDER)}\label{tab:cate_summary}
  \begin{tabular}{lrrrrr}
    	oprule
    Regime & P10 & Median & P90 & IQR & N Segments \\
    \midrule
  Low Inflation & \textemdash{} & \textemdash{} & \textemdash{} & \textemdash{} & \textemdash{} \\
  High Inflation & \textemdash{} & \textemdash{} & \textemdash{} & \textemdash{} & \textemdash{} \\
  Tight Labor & \textemdash{} & \textemdash{} & \textemdash{} & \textemdash{} & \textemdash{} \\
  Slack Labor & \textemdash{} & \textemdash{} & \textemdash{} & \textemdash{} & \textemdash{} \\
    \bottomrule
  \end{tabular}
\end{table}

\begin{table}[H]
  \centering
  \caption{Causal Forest Feature Importance (PLACEHOLDER)}\label{tab:feature_importance_cf}
  \begin{tabular}{lrr}
    	oprule
    Feature & Importance & Rank \\
    \midrule
  Feature A & \textemdash{} & 1 \\
  Feature B & \textemdash{} & 2 \\
  Feature C & \textemdash{} & 3 \\
  Feature D & \textemdash{} & 4 \\
  Feature E & \textemdash{} & 5 \\
    \bottomrule
  \end{tabular}
\end{table}

\begin{table}[H]
  \centering
  \caption{DML Feature Importance (PLACEHOLDER)}\label{tab:feature_importance_dml}
  \begin{tabular}{lrr}
    	oprule
    Feature & Importance & Rank \\
    \midrule
  Feature A & \textemdash{} & 1 \\
  Feature B & \textemdash{} & 2 \\
  Feature C & \textemdash{} & 3 \\
  Feature D & \textemdash{} & 4 \\
  Feature E & \textemdash{} & 5 \\
    \bottomrule
  \end{tabular}
\end{table}

\begin{table}[H]
  \centering
  \caption{Extended Policy Scenario Comparison (PLACEHOLDER)}\label{tab:policy_extended}
  \begin{tabular}{lrrrr}
    	oprule
    Scenario & Output Loss & Inflation Delta & Unemployment Delta & Welfare Index \\
    \midrule
  Baseline Increase & \textemdash{} & \textemdash{} & \textemdash{} & \textemdash{} \\
  Phased Increase & \textemdash{} & \textemdash{} & \textemdash{} & \textemdash{} \\
  Temporary Rebate & \textemdash{} & \textemdash{} & \textemdash{} & \textemdash{} \\
  Dual Adjustment & \textemdash{} & \textemdash{} & \textemdash{} & \textemdash{} \\
    \bottomrule
  \end{tabular}
\end{table}

\begin{table}[H]
  \centering
  \caption{Placebo and Overlap Diagnostics (PLACEHOLDER)}\label{tab:placebo_overlap}
  \begin{tabular}{lrrrr}
    	oprule
    Test & Rejection Rate & Mean Overlap & Tail PS Density & Comment \\
    \midrule
  Placebo (Pre-Policy) & \textemdash{} & \textemdash{} & \textemdash{} & \textemdash{} \\
  Synthetic Shock & \textemdash{} & \textemdash{} & \textemdash{} & \textemdash{} \\
  Trimmed Sample & \textemdash{} & \textemdash{} & \textemdash{} & \textemdash{} \\
  Reweighted Sample & \textemdash{} & \textemdash{} & \textemdash{} & \textemdash{} \\
    \bottomrule
  \end{tabular}
\end{table}
