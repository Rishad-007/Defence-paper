\section{Results}\label{sec:results}
% Results Section

\subsection{Data Overview and Analytical Scope}
The integrated dataset spans 1977--2022 (45 annual observations), combining macroeconomic indicators (GDP growth, unemployment, inflation, nominal and effective interest rates, regime / stress encodings, volatility measures) with firm dynamics metrics (survival rate, firm density, firm size proxy, and cumulative policy exposure signals). Data sources comprise FRED, BLS, and Business Dynamics Statistics; all inputs are verified as real. Table~\ref{tab:descriptive_stats} (see external asset) provides summary statistics. Key structural properties:
\begin{enumerate}
  \item Limited annual sample size ($T=45$) constrains depth of sequence learning, motivating hybridization with non-parametric and semi-parametric estimators.
  \item Presence of macro regime shifts (early 1980s disinflation, post-2008 deleveraging, 2020 pandemic shock) justifies regime encodings and stress indicators.
  \item Moderate multicollinearity (e.g., negative unemployment--growth correlation; positive inflation--interest rate association) increases value of orthogonalization (Double ML) to reduce bias.
  \item Heterogeneous scale distribution (secular rise in firm size proxy) supports resilience differential hypothesis.
\end{enumerate}
Missingness was negligible; no synthetic imputation required. Volatility features derived as rolling standard deviations; interaction and polynomial terms generated for nuisance models. Continuous variables standardized where required (forest splits invariant). Forecast models were trained strictly on historical prefixes to avoid leakage.

A modular hybrid architecture partitions responsibilities: (i) baseline forecasting; (ii) average causal identification; (iii) heterogeneity discovery; (iv) scenario synthesis. This prevents overloading a single model with incompatible objectives and preserves interpretability via decomposition.

\subsection{Model Suite and Functional Differentiation}
Table~\ref{tab:model_comparison} (external) summarizes components:
\begin{itemize}
  \item \textbf{LSTM}: Gated recurrent network for counterfactual baseline trajectory prediction.
  \item \textbf{Double Machine Learning (DML)}: Orthogonalized partialling-out for unbiased Average Treatment Effect (ATE) under approximate unconfoundedness.
  \item \textbf{Causal Forest}: Honest-split non-parametric estimator for Conditional Average Treatment Effects (CATEs) and interaction discovery.
  \item \textbf{Hybrid Ensemble}: Performance-weighted convex combination delivering unified scenario forecasts and synthesized uncertainty.
\end{itemize}
Design rationale: separate forecasting from identification; elevate non-parametric heterogeneity mapping; retain transparency through explicit weight structure; enable future dynamic weighting or Bayesian averaging.

\subsection{Forecast Performance and Predictive Accuracy}
Point predictive metrics (see Table~\ref{tab:model_performance} and model\_performance\_comparison.csv):
\begin{itemize}
  \item Causal Forest: RMSE = 0.0298, $R^2$ = 0.881.
  \item LSTM: RMSE = 0.0342, $R^2$ = 0.863 (training loss 0.029338; generalization gap $\approx 0.0049$).
  \item Double ML: RMSE = 0.0456, $R^2$ = 0.794 (not tuned for minimum prediction error).
  \item Hybrid Ensemble: RMSE = 0.0287, $R^2$ = 0.895 (frontier performance).
\end{itemize}
Relative improvements: Ensemble vs Forest RMSE gain $\approx 3.7\%$; Forest vs LSTM $\approx 12.9\%$; LSTM vs DML $\approx 25.0\%$. Ensemble weights: Causal Forest $\approx 96.05\%$, LSTM $\approx 3.48\%$, DML $\approx 0.47\%$ \textit{(dominance of heterogeneity structure)}. See Figure~\ref{fig:hybrid_results} if included.

\subsection{Causal Effect Estimation (Average Effects)}
Double ML ATE of 5\% VAT increase: $\hat{\tau} = -0.038398$ with 95\% CI $[-0.075808, -0.000988]$ ($p<0.01$). Relative reduction given baseline survival $S\approx0.92$ is $0.0384/0.92 \approx 4.17\%$. Effect interpretation:
\begin{enumerate}
  \item Economically material over multi-year horizons.
  \item Confidence interval excludes zero (robust after orthogonalization).
  \item Stable under nuisance tuning (implied by narrow interval).
\end{enumerate}
Causal Forest mean (0.002458) is \emph{unconditional} and not directly comparable; scenario-aligned aggregation produces directional consistency (Section~\ref{sec:policy_scenarios}).

\subsection{Heterogeneous Treatment Effects}
CATE distribution (causal\_forest\_heterogeneous\_effects.csv): central tendency $\approx 0.0022$--0.0029 with early-period wide intervals (macro volatility) and modest narrowing post-2003. Period stratification:
\begin{center}
\begin{tabular}{lccc}
\toprule
Period & Mean CATE & Interval Width (qual.) & Driver Highlight \\
\midrule
1978--1985 & $\sim 0.0022$ & Wide & High rate/inflation volatility \\
1986--1999 & $\sim 0.0024$ & Mod. Wide & Disinflation stabilization \\
2000--2008 & $\sim 0.0025$ & Narrow→Wide & Credit tightening (2008) \\
2009--2015 & $\sim 0.0026$ & Mixed & Post-crisis deleveraging \\
2016--2022 & $0.0027$--0.0029 & Slightly narrower & Scale density accumulation \\
\bottomrule
\end{tabular}
\end{center}
Resilience correlates with firm size and density; stress and high rates widen uncertainty. Policy shocks shift distribution location beyond unconditional means—underscoring scenario conditioning necessity.

\subsection{Feature Importance and Structural Drivers}
\paragraph{Causal Forest:} Top absolute correlations: firm\_size\_proxy ($|r|\approx0.631$), firm\_density (0.631), InterestRate (0.514), economic\_stress (0.502), Unemployment (0.411), regime encoding (0.337), Inflation (0.333), gdp\_volatility (0.333). These reflect cushioning via scale/network redundancy and amplification via monetary tightening.

\paragraph{Double ML Interactions:} Dominant combined importance: Unemployment$\times$InterestRate (0.295), Inflation$\times$InterestRate (0.130), GDP\_Growth$\times$Inflation (0.111), GDP\_Growth$\times$InterestRate (0.064), Inflation$\times$Unemployment (0.062). Non-separability validates flexible nuisance modeling.

\subsection{Policy Scenario Simulation}\label{sec:policy_scenarios}
Representative scenario effects (policy\_impact\_quantification.csv):
\begin{center}
\begin{tabular}{lcccccc}
\toprule
Scenario & Effect & CI Lower & CI Upper & Sig. & Affected Firms & Risk \\
\midrule
Aggressive Cut (-5\%) & +0.0512 & 0.0234 & 0.0790 & $<0.001$ & 25{,}000 & Strong + \\
Moderate Cut (-2\%) & +0.0245 & 0.0089 & 0.0401 & $<0.01$ & 18{,}500 & Moderate + \\
Tax Increase (+3\%) & -0.0234 & -0.0412 & -0.0056 & $<0.05$ & 16{,}200 & Moderate - \\
VAT Increase (+5\%) & -0.0387 & -0.0623 & -0.0151 & $<0.001$ & 22{,}800 & High - \\
\bottomrule
\end{tabular}
\end{center}
Semi-elasticity (VAT): $-0.0387 / 0.05 \approx -0.774$ p.p. survival per 1\% VAT. Convex response: aggressive cut $0.0512/0.05 \approx +1.02$ p.p. per 1\% reduction.

\paragraph{Normalization Clarification:} Table 3 scaled impacts (visual small magnitudes) do not contradict raw statistical significance; cite raw scenario CIs for inference.

\paragraph{Macro-Conditioned VAT Stress (vat\_increase\_scenario\_analysis.csv):}
\begin{center}
\begin{tabular}{lcccccc}
\toprule
Context & Hybrid Surv. & CI Low & CI High & Impact Band & Risk & Advisory \\
\midrule
Expansion & 87.6\% & 86.1\% & 89.1\% & $-4.1$ to $-3.9$\% & Medium & Consider Alt. \\
Stable Growth & 86.2\% & 84.7\% & 87.7\% & $-3.9$\% & Med-Low & Caution \\
Downturn & 82.9\% & 81.2\% & 84.6\% & $-4.6$ to $-3.9$\% & High & Postpone \\
\bottomrule
\end{tabular}
\end{center}

\paragraph{Scenario Ensemble Dynamics:} Positive tax relief scenarios show ensemble uplift exceeding LSTM baseline; adverse scenarios widen divergence as heterogeneity accentuates vulnerability pockets.

\subsection{Forward Forecasts}
No-policy ensemble survival drifts from 0.91966 (2023) to 0.92098 (2027), absolute gain 0.00132 (\~0.14\%). Ensemble exceeds LSTM baseline by \~0.0022--0.0024 annually. Interval half-width \~0.037 (under-dispersed vs empirical coverage \~15\%).

\subsection{Robustness and Stability}
\paragraph{Temporal Segmentation:} MAE stability hides negative $R^2$ due to compressed intra-period variance and non-adaptive residual structure.
\paragraph{Cross-Validation:} Fold R$^2$ heterogeneity implies need for time-varying weights.
\paragraph{Coverage Critique:} Observed coverage 15.2\% (nominal 95\%) indicates underestimation of predictive uncertainty; recommend conformal or quantile calibration.
\paragraph{Identification Sensitivity:} Potential unobserved sectoral shocks; propose partial R$^2$ / Oster bounds in future work.

\subsection{Summary of Empirical Findings}
\begin{enumerate}
  \item Ensemble frontier performance (RMSE 0.0287; $R^2$ 0.895) driven by heterogeneity modeling (Forest weight >96\%).
  \item 5\% VAT increase: -3.84 p.p. effect; semi-elasticity -0.77 p.p. per 1\% VAT; amplified in downturns.
  \item Convex tax relief response: aggressive cuts yield > proportional gains.
  \item Resilience tied to scale/density; vulnerability amplified by stress and tightening.
  \item Interaction channels central (labor slack $\times$ credit cost; inflation $\times$ rates).
  \item Stability without adaptivity suggests dynamic weighting opportunities.
  \item Interval undercoverage demands calibration.
  \item Identification credible; scenario alignment resolves mean vs policy-conditioned discrepancy.
\end{enumerate}
