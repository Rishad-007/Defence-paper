\section{Conclusion}\label{sec:conclusion}
This thesis introduced a hybrid machine learning and econometric framework for evaluating the macroeconomic and distributional impacts of \VAT{} policy interventions. By integrating \LSTM{} sequence modeling with causal forest heterogeneity estimation and robust validation procedures, the framework improves predictive fidelity and policy interpretability relative to conventional baselines.

Key findings include: (i) enhanced multi-horizon forecast accuracy; (ii) identification of inflation-regime-dependent treatment amplification; (iii) stable aggregate ATE estimates with interpretable CATE dispersion; and (iv) actionable scenario simulations supporting risk-aware fiscal decision-making.

The approach advances methodological synthesis in policy analytics by embedding dynamic state representations into causal effect estimation while maintaining diagnostic transparency. Limitations—such as residual confounding risks and regime sparsity—motivate further refinement.

Future research may extend the framework to multi-policy interaction analysis, incorporate structural equilibrium constraints, and explore adaptive experimentation for real-time policy calibration. The results underscore the promise of hybrid analytics in modern fiscal governance.
