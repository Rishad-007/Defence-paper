% Conclusion Section
\section{Conclusion}\label{sec:conclusion}

\subsection{Summary of Core Contributions}
\begin{enumerate}
  \item \textbf{Framework Integration}: Unified forecasting, causal identification, heterogeneity mapping, and scenario simulation architecture.
  \item \textbf{Methodological Innovation}: Performance-weighted ensemble dominated by heterogeneity component with transparent contribution decomposition.
  \item \textbf{Empirical Insight}: Quantified adverse VAT semi-elasticity (\(-0.77\) p.p. survival per 1\% VAT) and convex relief response under macro conditioning.
  \item \textbf{Interpretability Layer}: Interaction-level explanation bridging econometric structure and machine learning flexibility.
  \item \textbf{Policy Workflow}: Scenario classification translating quantitative outputs into ordinal risk advisories.
\end{enumerate}

\subsection{Principal Empirical Insight}
A 5\% VAT increase induces an economically meaningful, macro-conditioned reduction in firm survival (\(-3.84\) p.p. baseline; larger in downturns). Aggressive tax reductions deliver more-than-proportional survival gains, evidencing convexity in fiscal response surfaces and informing countercyclical stabilization strategy.

\subsection{Theoretical and Practical Integration}
Findings operationalize financial accelerator, real options, and agglomeration externalities into measurable causal channels. Heterogeneity mapping enables precision targeting of transitional supports; scenario simulation equips fiscal governance with ex-ante risk triage capability.

\subsection{Policy Design Guidance}
\begin{enumerate}
  \item \textbf{Temporal Alignment}: Defer contractionary shifts in recessionary regimes; exploit expansions for structural tax adjustments.
  \item \textbf{Compensatory Coupling}: Pair VAT increases with liquidity or capital formation incentives to neutralize net survival drag.
  \item \textbf{Relief Non-Linearity}: Target thresholds where aggressive cuts yield convex resilience gains relative to marginal revenue cost.
  \item \textbf{Adaptive Triggers}: Embed macro-contingent clauses to automatically defer adverse measures under stress.
  \item \textbf{Monitoring Layer}: Integrate scenario outputs into continuous fiscal risk dashboards.
\end{enumerate}

\subsection{Methodological Reflections}
Dominance of heterogeneity modeling suggests marginal returns lie in dynamic weighting and calibrated uncertainty rather than deeper sequence complexity at annual resolution. Under-dispersed predictive intervals necessitate formal calibration (conformal, quantile modeling, or Bayesian ensembles). The architecture is composable—incremental upgrades preserve interpretability.

\subsection{Closing Statement}
The hybrid framework provides a scalable, interpretable blueprint for policy impact analytics under structural uncertainty, joining causal rigor with predictive sharpness and heterogeneity awareness. Future enhancements (uncertainty calibration, dynamic gating, sectoral stratification) can further refine precision policy design and resilience planning.
