\begin{abstract}
This study presents a comprehensive hybrid approach to economic policy analysis by integrating advanced machine learning techniques with traditional econometric models. Leveraging an extensive dataset of macroeconomic indicators spanning multiple years and sectors, the research develops, implements, and systematically compares the performance of models such as Long Short-Term Memory (LSTM) neural networks and causal forests. These models are employed to quantify and interpret the impact of policy interventions, with a particular emphasis on changes in value-added tax (VAT) rates and their broader economic consequences.

The hybrid analytical framework enables both robust time-series forecasting and rigorous causal inference, allowing for a detailed examination of direct, indirect, and heterogeneous effects of policy measures across different economic segments. The methodology incorporates cross-validation, feature importance analysis, and scenario-based simulations to ensure the reliability and interpretability of results. Empirical findings reveal that the combined use of machine learning and econometric techniques significantly enhances predictive accuracy, model stability, and the depth of policy evaluation compared to conventional approaches.

The study further provides actionable insights for policymakers by identifying key drivers of economic response to VAT changes and quantifying the distributional impacts on various sectors. The results highlight the value of hybrid modeling in addressing complex, real-world economic questions and demonstrate its potential to improve the reliability and effectiveness of policy impact assessments. This research contributes to the growing field of data-driven economic analysis and offers a scalable framework for future policy evaluation efforts.
\end{abstract}


\keywords{Causal machine learning, policy counterfactuals, VAT, Double~ML, LSTM, regret minimization}


