\begin{abstract}

Traditional approaches to evaluating the impacts of fiscal policies like Value-Added Tax (VAT) often rely on linear models that cannot fully capture the complex and varied effects on the economy. This limits our understanding of how VAT changes influence household consumption, business survival, and overall economic performance. VAT affects many aspects of daily economic life, from consumer prices to government revenues and firm-level outcomes. Accurately measuring these impacts is crucial for designing effective tax policies. To overcome the limitations of traditional methods, this study develops a hybrid framework that combines Long Short-Term Memory (LSTM) models for economic forecasting, Double Machine Learning (Double ML) for unbiased causal effect estimation, and Causal Forests to detect heterogeneous treatment effects. Using extensive data spanning multiple sectors and several decades, the framework provides robust forecasts and reliable causal insights. Our results show improved predictive accuracy (RMSE 0.0287; \( R^2 = 0.895 \)) and greater interpretability. Key findings indicate that a 5\% increase in VAT reduces economic output by approximately 3.8 percentage points (semi-elasticity of \(-0.77\) p.p. per 1\% VAT), while aggressive tax cuts produce relatively larger gains. The analysis also highlights that dense, large-scale sectors are more resilient, whereas vulnerabilities increase with monetary tightening. These findings demonstrate the value of integrating advanced machine learning techniques with econometric methods, providing more actionable and trustworthy policy evaluations in complex economic environments.

\end{abstract}





\keywords{Value-Added Tax, Machine Learning, Causal Inference, LSTM, Causal Forest, Econometric Models, Economic Forecasting, Firm Survival, Policy Impact}



