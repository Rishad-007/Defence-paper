\begin{abstract}
This thesis proposes a hybrid causal machine learning framework for evaluating the counterfactual impact of economic policies across both macroeconomic and microeconomic dimensions. Traditional econometric approaches often struggle with high-dimensional, dynamic, and heterogeneous data, limiting their ability to provide reliable insights for complex policy decisions. To address this, we integrate multiple causal ML models---LSTM-based time series forecasting for economic indicators, Double Machine Learning for isolating treatment effects, Causal Forests for modeling treatment heterogeneity, and NLP-based tools for extracting insights from unstructured policy text.

Our methodology follows a four-stage pipeline: (1) Causal Discovery to uncover structural dependencies between variables; (2) Estimation using robust ML-based causal inference models; (3) Validation through out-of-sample and placebo testing; and (4) Simulation for forecasting policy outcomes under alternative scenarios. This approach is applied to real-world datasets involving fiscal and economic indicators, producing both aggregate-level and individual-level policy insights.

As a case study, we simulate the effects of a hypothetical tax policy adjustment to illustrate the model's capability in quantifying heterogeneous impacts and potential policy regret. The results show that hybrid causal ML methods outperform traditional inference techniques in robustness, granularity, and decision relevance. This framework contributes a flexible, modular approach to policy evaluation, offering actionable insights for data-driven economic governance.
\end{abstract}

\keywords{Causal machine learning, policy counterfactuals, VAT, Double~ML, LSTM, regret minimization}


